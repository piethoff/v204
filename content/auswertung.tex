\section{Auswertung}
\label{sec:Auswertung}
\subsection{Statische Methode}
\begin{figure}
    \centering
    \includegraphics{plot.pdf}
    \caption{Temperaturverläufe der äußeren Thermoelemente.}
    \label{fig:plot}
\end{figure}
Die Temperaturverläufe der äußeren Thermoelemente wurden in \ref{fig:plot} aufgetragen.
Auffallend ist, dass alle Graphen nach ungefähr 200 Abstraten ein Plateau aufweisen.
Die Temperatur der beiden Messingstäbe steigt jedoch langsamer an als die des Aluminiumstabes und schneller als die Temperatur des Edelstahlstabes.
%
Nach 700 Sekunden lässt sich dies schon beobachten.
Dort besitzen sie die follgenden Temperaturen:
\begin{table}
    \centering
    \caption{Temperaturen nach 700 Sekunden.}
    \label{tab:t1}
    \sisetup{table-format=1.2}
    \begin{tabular}{S S}
        \toprule
        {$T$} & {Temperatur$/\si{\celsius}$} \\
        \midrule
        1 & 27.5 \\
        4 & 26.5 \\
        5 & 28.5 \\
        8 & 24 \\
        \bottomrule
    \end{tabular}
\end{table}
\begin{figure}
    \centering
    \includegraphics{plot2.pdf}
    \caption{Temperaturdifferenz der Thermoelemente an Messing und Edelstahl.}
    \label{fig:plot2}
\end{figure}

In \ref{fig:plot2} ist die Temperaturdifferenz nach der Zeit aufgetragen.
Die jeweiligen Graphen beziehen sich auf Messing- und Edelstahlstab.
Die Differenz steigt bei beiden erst rapide an.
Bei dem Messingstab fällt die kurve jedoch wieder stark ab und verläuft im späteren Verlauf nahezu waagerecht.
Die Temperaturdifferenz der Thermoelemente des Edelstahlstabes fällt nur langsam nach dem Maximum ab.

\subsection{Dynamische Methode}
\begin{figure}
    \centering
    \includegraphics{plot3.pdf}
    \caption{Temperaturverläufe des breiten Messingstabs.}
    \label{fig:plot3}
\end{figure}
Zunächst soll der breite Messingstab betrachtet werden.
In \ref{fig:plot3} ist die Temperatur an den entsprechenden Messpunkten gegen die Zeit aufgetragen. Aus der Graphik kann ist abzulesen, 
dass eine Phasenverschiebung $\symup{\Delta} t$ von $\SI{999999999}{\second}$ vorliegt. die Amplituden der Temperaturwellen liegen bei 
$A_1 = \SI{9999999}{\kelvin}$ und $A_2 = \SI{9999999}{\kelvin}$.
Für die Wärmeleitfähigkeit $\kappa$ folgt nach Gleichung \eqref{eqn:wleitfaehigkeit}, $\kappa_\text{Messing} = 
\SI{9999999}{\watt\per\meter\per\kelvin}$.
\begin{figure}
    \centering
    \includegraphics{plot4.pdf}
    \caption{Temperaturverläufe des Aluminiumstabs.}
    \label{fig:plot4}
\end{figure}

Für den Aluminiumstab ergeben sich ähnliche Temperaturverläufe, in \ref{fig:plot4} gegen die Zeit aufgetragen. Auch hier können die 
Phasenverschiebung $\symup{\Delta} t = \SI{99999}{\second}$ und die Amplituden $A_5 = \SI{999}{\kelvin}$ und $A_6 = \SI{999}{\kelvin}$ 
abgelesen werden. Es ergibt sich aus Gleichung \eqref{eqn:wleitfaehigkeit}, $\kappa_\text{Aluminium} = \SI{999}{\watt\per\meter\kelvin}$.

In der zweiten Messung wird, bei veränderter Periode, die Wärmeleitfähigkeit des Edelstahlstabs bestimmt.
\begin{figure}[H]
    \centering
    \includegraphics{plot5.pdf}
    \caption{Temperaturverläufe des Edelstahlstabs.}
    \label{fig:plot5}
\end{figure}
Aus der Graphik lassen sich $\symup{\Delta} t = \SI{987}{\second}$, $A_7 = \SI{98798}{\kelvin}$ und $A_8 = \SI{6541}{\kelvin}$ ermitteln.
Es folgt aus GLeichung \eqref{eqn:wleitfaehigkeit} $\kappa_\text{Edelstahl} = \SI{35163}{\watt\per\meter\kelvin}$.

Zusammenfassend sind die ermittelten Wärmeleitfähigkeiten \ref{tab:t2}:
\begin{table}[H]
    \centering
    \caption{Wärmeleitfähigkeiten.}
    \label{tab:t2}
    \sisetup{table-format=1.2}
    \begin{tabular}{c S}
        \toprule
        {Material}  & {$\kappa/\si[per-mode=reciprocal]{\watt\per\meter\kelvin}$} \\
        \midrule
        Messing     & 999.45\\
        Aluminium   & 999   \\
        Edelstahl   & 999.2 \\
        \bottomrule
    \end{tabular}
\end{table}
