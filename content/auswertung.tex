\section{Auswertung}
\label{sec:Auswertung}
\subsection{Statische Methode}
\begin{figure}
  \centering
  \includegraphics{plot.pdf}
  \caption{Plot.}
  \label{fig:plot}
\end{figure}
Die Temperaturverläufe der äußeren Thermoelemente wurden in \ref{fig:plot} aufgetragen.
Auffallend ist, dass alle Graphen nach ungefähr 200 Abstraten ein Plateau aufweisen.
Die Temperatur der beiden Messingstäbe steigt jedoch langsamer an als die des Aluminiumstabes und schneller als die Temperatur des Edelstahlstabes.
Nach 700 Sekunden lässt sich dies schon beobachten.
Dort besitzen sie die follgenden Temperaturen:
\begin{table}
  \centering
  \caption{Temperaturen nach 700 Sekunden}
    \label{tab:t_1}
    \begin{tabular}{c c}
      \toprule
       $T$ & "Temperatur[Celsius]" \\
      \midrule
      1 & 27.5 \\
      4 & 26.5 \\
      5 & 28.5 \\
      8 & 24 \\
      \bottomrule
    \end{tabular}
\end{table}
