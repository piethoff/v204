\section{Auswertung}
\label{sec:Auswertung}
\subsection{Statische Methode}
\begin{figure}
  \centering
  \includegraphics{plot.pdf}
  \caption{Temperaturverläufe der äußeren Thermoelemente}
  \label{fig:plot}
\end{figure}
Die Temperaturverläufe der äußeren Thermoelemente wurden in \ref{fig:plot} aufgetragen.
Auffallend ist, dass alle Graphen nach ungefähr 200 Abstraten ein Plateau aufweisen.
Die Temperatur der beiden Messingstäbe steigt jedoch langsamer an als die des Aluminiumstabes und schneller als die Temperatur des Edelstahlstabes.


Nach 700 Sekunden lässt sich dies schon beobachten.
Dort besitzen sie die follgenden Temperaturen:
\begin{table}
  \centering
  \caption{Temperaturen nach 700 Sekunden}
    \label{tab:t_1}
    \begin{tabular}{c c}
      \toprule
       $T$ & "Temperatur[Celsius]" \\
      \midrule
      1 & 27.5 \\
      4 & 26.5 \\
      5 & 28.5 \\
      8 & 24 \\
      \bottomrule
    \end{tabular}
\end{table}
\begin{figure}
  \centering
  \includegraphics{plot2.pdf}
  \caption{Temperaturdifferenz der Thermoelemente an Messing und Edelstahl}
  \label{fig:plot2}
\end{figure}


In \ref{fig:plot2} ist die Temperaturdifferenz nach der Zeit aufgetragen.
Die jeweiligen Graphen beziehen sich auf Messing- und Edelstahlstab.
Die Differenz steigt bei beiden erst rapide an.
Bei dem Messingstab fällt die kurve jedoch wieder stark ab und verläuft im späteren Verlauf nahezu waagerecht.
Die Temperaturdifferenz der Thermoelemente des Edelstahlstabes fällt nur langsam nach dem Maximum ab.
