\section{Durchführung}
\label{sec:Durchführung}
Die verscheidenen Wärmeleitfähigkeiten werden mithilfe des folgenden Versuchsaufbaus \ref{fig:aufbau}
untersucht. Die Thermoelemente haben jeweils den Abstand \SI{30}{\milli\meter}.
\begin{figure}
    \centering
    \includegraphics[width=\textwidth]{content/aufbau.png}
    \caption{Versuchsaufbau\cite{v204}}
    \label{fig:aufbau}
\end{figure}
Dabei heizt das Peltierelement die vier Metallstäbe simultan und die Temperatur wird an 8
verschiedenen Messpunkten, je zwei pro Stab, mithilfe vom Thermoelementen gemessen. Die
Messwerte werden über ein Temperature Array mit einem Datenlogger aufgezeichnet.
Die Maße der zu untersuchenden Metallstäbe sind:
\begin{table}
    \centering
    \caption{Abmessungen der Metallstäbe \cite{v204}}
    \label{tab:werte}
    \sisetup{table-format=1.2}
    \begin{tabular}{c c S S S}
        \toprule
        {Material} & {Abmessungen$/\si{\centi\meter}$} &{$\rho/\si{\kg\per\meter\cubed}$} &{$c/\si{\joule\per\kg\per\kelvin}$} &{$\kappa/\si{\watt\per\meter\kelvin}$\cite{waermeleit}} \\
        \midrule
        Messing (breit) & 9 x 1.2 x 0.4 & 8520 & 385 & 120 \\
        Messing (schmal) & 9 x 0.7 x 0.4 & 8520 & 385 & 120 \\
        Aluminium (breit) & 9 x 1.2 x 0.4 & 2800 & 830 & 237 \\
        Edelstahl (breit) & 9 x 1.2 x 0.4 & 8000 & 400 & 15 \\
        \bottomrule
    \end{tabular}
\end{table}
%
\subsection{Statische Methode}
\label{sec:statische Methode}
Das Peltierelement wird mit $U_\text{P}=\SI{5}{\volt}$ betrieben.
Die vier Metallstäbe werden eine halbe Stunde lang erhitzt.
An den acht Thermoelementen wird die Temperatur gemessen und der jeweilige Wert alle 5 Sekunden gespeichert.
Anschließend werden die Stäbe durch das Peltierelement gekühlt.
%
\subsection{Dynamische Methode}
\label{sec:dynamische Methode}
Bei der dynamischen Methode, auch Angström-Verfahren, wird im Gegensatz zur statischen Methode nicht kontinuierlich,
sondern periodisch mit $U_\text{P}=\SI{8}{\volt}$ geheizt und gekühlt.
Der Datenlogger wird außerdem auf eine Abtastperiode von $\SI{2}{\second}$ gestellt.
Es ergibt sich innerhalb des Materials eine Temperaturwelle wie durch Gleichung \eqref{eqn:thermowelle} beschrieben.
In der ersten Messung werden zehn Perioden mit einer Periodendauer von $\SI{80}{\second}$ aufgezeichnet,
in der zweiten Messung ist etwa eine Stunde lang mit einer Periodendauer von $\SI{400}{\second}$ gemessen.
