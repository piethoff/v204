\section{Diskussion}
\label{sec:Diskussion}
Anhand der Information die aus der statischen Messmethode gewonnen werden, lässt sich eine erste Abschätzung über die Wärmeleitfähigkeit der verschiedenen Materialien machen.
Nach 700 Sekunden ist bereits ersichtlich, dass sich der Aluminiumstab am schnellsten erwärmt.
Folglich sollter dieser die größte Wärmeleitfähigkeit besitzen.
Am zweitschnellsten erwärmt sich Messing unabhängig von seinem Durchmesser.
Dabei ist jedoch zu beachten, dass der Wärmestrom stärker ist bei größerem Durchmesser.
Die Temperaturdifferenz bei Edelstahl ist größer als diese bei Messing, was auch durch die Wärmeleitfähigkeit bedingt ist.
Aus \ref{tab:3} wird erkenntlich, dass der Wärmestrom unabhangig von der Temperatur des Materials ist.

Mithilfe der dynamischen Methode kann die Wärmeleitfähigkeit nicht nur qualitativ, sondern auch quantitativ bestimmt werden.
Jedoch ist die dazu notwendige Bestimmung der Amplituden und Phasenverschiebungen durch Ablesen am Graphen sehr mühsam und fehlerbehaftet.
Die starken Abweichungen zu den Literaturwerten können einerseits durch die fehlerbehaftete 
Messmethode, andererseits aber auch dadurch erklärt werden, dass die genaue Legierung der 
Metalle nicht bekannt ist.
Durch die Angström-Methode lassen sich die Erkenntnisse der statischen Methode bestätigen und genauer konkretisieren.
