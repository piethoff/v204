\section{Zielsetzung}
Die Wärmeleitfähigkeit von Aluminium, Edelstahl und Messing soll untersucht werden.
\section{Theorie}
\label{sec:Theorie}
In einem Körper, welcher sich in einem Temperaturungleichgewicht befindet, kommt es zum Wärmetransport entlang des Temperaturgefälles.
Für Festkörper findet der Wärmetransport über Phononen und freie Elektronen statt.
Es wird ein Stab der Länge L und Querschnittsfläche A betrachtet.
Er besitzt die Materialdichte $\rho$ und spezifische Wärme $c$.
Die Wärmemenge,
\begin{equation}
  \label{eqn:gl1}
    \symup{d}Q =-  \kappa A \frac{\partial T}{\partial x} \symup{d}t,
\end{equation}
fließt in der Zeit $\symup{dt}$ durch die Querschnittsfläche $A$.
Das negative Vorzeichen deutet an, dass der Wärmestrom gemäß des zweiten Haupsatzes der
Thermodynamik in Richtung abnehmender Temperatur fließt.
Für die Wärmestromdichte gilt
\begin{equation}
    j_\omega =- \kappa \frac{\partial T}{\partial x}.
\end{equation}
Aus der Kontinuitätsgleichung ergibt sich die eindimensionale Wärmeleistungsgleichung
\begin{equation}
    \frac{\partial T}{\partial t} = \frac {\kappa}{\rho} \frac{\partial^2  T}{\partial x^2}.
\end{equation}
Sie beschreibt räumliche und zeitliche Entwicklung der Temperaturverteilung.
Die Temperaturleitfähigkeit
\begin{equation}
    \sigma_T = \frac{\kappa}{\rho c}
\end{equation}
gibt die Geschwindigkeit an mit welcher sich eine Temperaturdifferenz ausgleicht.
%
Eine Temperaturwelle der Form
\begin{equation}
    T\left(x,t\right) = T_\text{max}e^{\sqrt{\frac{\omega\rho c}{2 \kappa}}x} \cos\left(\omega t - \sqrt{\frac{\omega\rho c}{2 \kappa}}x\right)
\label{eqn:thermowelle}
\end{equation}
entsteht in einem sehr langen Stab durch periodischen Temperaturwechsel.
Ihre Phasengeschwindigkeit lautet:
\begin{equation}
    \upsilon = \sqrt{\frac{2\kappa\omega}{\rho c}}
\end{equation}
Für die Wärmeleitfähigkeit ergibt sich, mit  $\omega=\frac{2\pi}{T^*}$ und $\phi=\frac{2\pi\symup{\Delta}t}{T^*}$,
\begin{equation}
    \kappa = \frac{\rho c\left(\symup{\Delta}x\right)^2}{2 ln\left(A_\text{nah}/A_\text{fern}\right)\symup{\Delta}t}.
    \label{eqn:wleitfaehigkeit}
\end{equation}
Dabei ergibt sich die Dämpfung aus dem Amplitudenverhältnis $A_\text{nah}$ und $A_\text{fern}$ der Welle an zwei Messpunkten
$x_\text{nah}$ und $x_\text{fern}$.
$\symup{\Delta}t$ ist die Phasendifferenz der Welle zwischen den beiden Messpunkten und $\symup{\Delta}x$ ist der Abstand zwischen
diesen.
